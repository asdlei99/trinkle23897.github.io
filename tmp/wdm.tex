\documentclass{article}
\usepackage[utf8]{inputenc}
\usepackage{ctex}
\usepackage[linesnumbered,boxed]{algorithm2e}

\title{algorithm}
\author{n+e}
\date{July 2017}

\begin{document}

\maketitle

\section{Algorithm}
\begin{algorithm}
%\small
%\scriptsize
\KwIn{当前芯片的布线状态,其中靠近原点的子区域尚未布线}
\KwOut{在该子区域布线之后的芯片布线状态}
\While{子区域中连线起点尚未全部连接 $\land$ 有剩余连线终点}{
	\If{已经到达芯片角落}{
		退出该循环\;
	}
	\If{连线终点落后于连线起点}{
		选择$n_x$($n_y$)所在的那一列(行),左一个右一个地连出去,直到边界被连接的连线终点连续排列\;
	}
	Call function Link\_in to obtain $pt_{x_1},pt_{y_1}$,以及备选的两条线路\;
	计算$f_x,f_y$的值,如果某个值为1则置该侧线路长度为$+\infty$\;
	\If{某侧的$f$值不为$1\land$计算该侧线路失败}{\Return{该子区域无合法布线状态\;}}
	比较两条线路长度,将长度小的一条记录至布线状态中\;
	\If{另一条线路的起点如果连向对侧的长度比该线路要长}{将该线路也记录至布线状态中\;}
}
\If{子区域中连线起点尚未全部连接 $\land$ 无剩余连线终点}{\Return{该子区域无合法布线状态\;}}
\While{子区域中连线起点尚未全部连接}{
	\uIf{选取某一列全部连接至$x$轴的总长度要比选取某一行全部连接至$y$轴的总长度要短}{
		选取$n_x$所在的一列,将其全部连接至$x$轴\;
	}
	\Else{
		选取$n_y$所在的一行,将其全部连接至$y$轴\;
	}
}
\Return{当前芯片的布线状态}\;
\caption{子区域中的布线算法.}
\label{alg:dc}
\end{algorithm}
\begin{algorithm}
%\small
%\scriptsize
\KwIn{The length $node_x$ and width $node_y$ of the biochip.}
\KwOut{最小的间隔$d$、布线总长度和布线方案}
\If{$\min\{N,M\}\le 2$}{
    直接构造出方案并返回\;
}
将连线起点划分成4个子区域\;
确定$d$的上下界,使用二分法判定在该状态下是否有可行方案\;
\ForEach{二分出来的 $d$}
{
	Calculating total length and width of biochip\;
	Connecting origins near central axis\;
	\For{每个子区域}{
		Call Algorithm~\ref{alg:dc} to obtain 子区域中的布线方案 with 当前芯片布线状态 as input\;
		\If{该函数返回布线失败}{
			$d$不合法,退出该循环\;
		}
		绕坐标轴进行坐标变换\;
	}
	计算下一个需要判断的$d$\;
}
\Return{最小的间隔$d$、布线总长度和布线方案}\;
\caption{Complete routing method of RRBI.????}
\label{alg:rule}
\end{algorithm}
\end{document}
